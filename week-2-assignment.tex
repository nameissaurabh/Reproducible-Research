\documentclass[]{article}
\usepackage{lmodern}
\usepackage{amssymb,amsmath}
\usepackage{ifxetex,ifluatex}
\usepackage{fixltx2e} % provides \textsubscript
\ifnum 0\ifxetex 1\fi\ifluatex 1\fi=0 % if pdftex
  \usepackage[T1]{fontenc}
  \usepackage[utf8]{inputenc}
\else % if luatex or xelatex
  \ifxetex
    \usepackage{mathspec}
  \else
    \usepackage{fontspec}
  \fi
  \defaultfontfeatures{Ligatures=TeX,Scale=MatchLowercase}
\fi
% use upquote if available, for straight quotes in verbatim environments
\IfFileExists{upquote.sty}{\usepackage{upquote}}{}
% use microtype if available
\IfFileExists{microtype.sty}{%
\usepackage[]{microtype}
\UseMicrotypeSet[protrusion]{basicmath} % disable protrusion for tt fonts
}{}
\PassOptionsToPackage{hyphens}{url} % url is loaded by hyperref
\usepackage[unicode=true]{hyperref}
\hypersetup{
            pdftitle={Reproducible Research Assignment W2},
            pdfauthor={Saurabh Ghadge},
            pdfborder={0 0 0},
            breaklinks=true}
\urlstyle{same}  % don't use monospace font for urls
\usepackage[margin=1in]{geometry}
\usepackage{color}
\usepackage{fancyvrb}
\newcommand{\VerbBar}{|}
\newcommand{\VERB}{\Verb[commandchars=\\\{\}]}
\DefineVerbatimEnvironment{Highlighting}{Verbatim}{commandchars=\\\{\}}
% Add ',fontsize=\small' for more characters per line
\usepackage{framed}
\definecolor{shadecolor}{RGB}{248,248,248}
\newenvironment{Shaded}{\begin{snugshade}}{\end{snugshade}}
\newcommand{\KeywordTok}[1]{\textcolor[rgb]{0.13,0.29,0.53}{\textbf{#1}}}
\newcommand{\DataTypeTok}[1]{\textcolor[rgb]{0.13,0.29,0.53}{#1}}
\newcommand{\DecValTok}[1]{\textcolor[rgb]{0.00,0.00,0.81}{#1}}
\newcommand{\BaseNTok}[1]{\textcolor[rgb]{0.00,0.00,0.81}{#1}}
\newcommand{\FloatTok}[1]{\textcolor[rgb]{0.00,0.00,0.81}{#1}}
\newcommand{\ConstantTok}[1]{\textcolor[rgb]{0.00,0.00,0.00}{#1}}
\newcommand{\CharTok}[1]{\textcolor[rgb]{0.31,0.60,0.02}{#1}}
\newcommand{\SpecialCharTok}[1]{\textcolor[rgb]{0.00,0.00,0.00}{#1}}
\newcommand{\StringTok}[1]{\textcolor[rgb]{0.31,0.60,0.02}{#1}}
\newcommand{\VerbatimStringTok}[1]{\textcolor[rgb]{0.31,0.60,0.02}{#1}}
\newcommand{\SpecialStringTok}[1]{\textcolor[rgb]{0.31,0.60,0.02}{#1}}
\newcommand{\ImportTok}[1]{#1}
\newcommand{\CommentTok}[1]{\textcolor[rgb]{0.56,0.35,0.01}{\textit{#1}}}
\newcommand{\DocumentationTok}[1]{\textcolor[rgb]{0.56,0.35,0.01}{\textbf{\textit{#1}}}}
\newcommand{\AnnotationTok}[1]{\textcolor[rgb]{0.56,0.35,0.01}{\textbf{\textit{#1}}}}
\newcommand{\CommentVarTok}[1]{\textcolor[rgb]{0.56,0.35,0.01}{\textbf{\textit{#1}}}}
\newcommand{\OtherTok}[1]{\textcolor[rgb]{0.56,0.35,0.01}{#1}}
\newcommand{\FunctionTok}[1]{\textcolor[rgb]{0.00,0.00,0.00}{#1}}
\newcommand{\VariableTok}[1]{\textcolor[rgb]{0.00,0.00,0.00}{#1}}
\newcommand{\ControlFlowTok}[1]{\textcolor[rgb]{0.13,0.29,0.53}{\textbf{#1}}}
\newcommand{\OperatorTok}[1]{\textcolor[rgb]{0.81,0.36,0.00}{\textbf{#1}}}
\newcommand{\BuiltInTok}[1]{#1}
\newcommand{\ExtensionTok}[1]{#1}
\newcommand{\PreprocessorTok}[1]{\textcolor[rgb]{0.56,0.35,0.01}{\textit{#1}}}
\newcommand{\AttributeTok}[1]{\textcolor[rgb]{0.77,0.63,0.00}{#1}}
\newcommand{\RegionMarkerTok}[1]{#1}
\newcommand{\InformationTok}[1]{\textcolor[rgb]{0.56,0.35,0.01}{\textbf{\textit{#1}}}}
\newcommand{\WarningTok}[1]{\textcolor[rgb]{0.56,0.35,0.01}{\textbf{\textit{#1}}}}
\newcommand{\AlertTok}[1]{\textcolor[rgb]{0.94,0.16,0.16}{#1}}
\newcommand{\ErrorTok}[1]{\textcolor[rgb]{0.64,0.00,0.00}{\textbf{#1}}}
\newcommand{\NormalTok}[1]{#1}
\usepackage{graphicx,grffile}
\makeatletter
\def\maxwidth{\ifdim\Gin@nat@width>\linewidth\linewidth\else\Gin@nat@width\fi}
\def\maxheight{\ifdim\Gin@nat@height>\textheight\textheight\else\Gin@nat@height\fi}
\makeatother
% Scale images if necessary, so that they will not overflow the page
% margins by default, and it is still possible to overwrite the defaults
% using explicit options in \includegraphics[width, height, ...]{}
\setkeys{Gin}{width=\maxwidth,height=\maxheight,keepaspectratio}
\IfFileExists{parskip.sty}{%
\usepackage{parskip}
}{% else
\setlength{\parindent}{0pt}
\setlength{\parskip}{6pt plus 2pt minus 1pt}
}
\setlength{\emergencystretch}{3em}  % prevent overfull lines
\providecommand{\tightlist}{%
  \setlength{\itemsep}{0pt}\setlength{\parskip}{0pt}}
\setcounter{secnumdepth}{0}
% Redefines (sub)paragraphs to behave more like sections
\ifx\paragraph\undefined\else
\let\oldparagraph\paragraph
\renewcommand{\paragraph}[1]{\oldparagraph{#1}\mbox{}}
\fi
\ifx\subparagraph\undefined\else
\let\oldsubparagraph\subparagraph
\renewcommand{\subparagraph}[1]{\oldsubparagraph{#1}\mbox{}}
\fi

% set default figure placement to htbp
\makeatletter
\def\fps@figure{htbp}
\makeatother


\title{Reproducible Research Assignment W2}
\author{Saurabh Ghadge}
\date{December 5, 2021}

\begin{document}
\maketitle

\paragraph{1) Reading Data}\label{reading-data}

\begin{Shaded}
\begin{Highlighting}[]
\NormalTok{data <-}\StringTok{ }\KeywordTok{read.csv}\NormalTok{(}\StringTok{"repdata_data_activity/activity.csv"}\NormalTok{)}
\KeywordTok{head}\NormalTok{(data,}\DecValTok{5}\NormalTok{)}
\end{Highlighting}
\end{Shaded}

\begin{verbatim}
##   steps       date interval
## 1    NA 2012-10-01        0
## 2    NA 2012-10-01        5
## 3    NA 2012-10-01       10
## 4    NA 2012-10-01       15
## 5    NA 2012-10-01       20
\end{verbatim}

\paragraph{2)Histogram of the total number of steps taken each
day}\label{histogram-of-the-total-number-of-steps-taken-each-day}

\begin{Shaded}
\begin{Highlighting}[]
\KeywordTok{library}\NormalTok{(tidyverse)}
\end{Highlighting}
\end{Shaded}

\begin{verbatim}
## Warning: package 'tidyverse' was built under R version 4.1.1
\end{verbatim}

\begin{verbatim}
## -- Attaching packages --------------------------------------- tidyverse 1.3.1 --
\end{verbatim}

\begin{verbatim}
## v ggplot2 3.3.5     v purrr   0.3.4
## v tibble  3.1.2     v dplyr   1.0.7
## v tidyr   1.1.3     v stringr 1.4.0
## v readr   2.0.0     v forcats 0.5.1
\end{verbatim}

\begin{verbatim}
## -- Conflicts ------------------------------------------ tidyverse_conflicts() --
## x dplyr::filter() masks stats::filter()
## x dplyr::lag()    masks stats::lag()
\end{verbatim}

\begin{Shaded}
\begin{Highlighting}[]
\NormalTok{data2 <-}\StringTok{ }\NormalTok{data }\OperatorTok\StringTok{ }\KeywordTok{group_by}\NormalTok{(date) }\OperatorTok\StringTok{ }\KeywordTok{summarize}\NormalTok{(}\DataTypeTok{tot_steps =} \KeywordTok{sum}\NormalTok{(steps,}\DataTypeTok{na.rm=}\OtherTok{TRUE}\NormalTok{))}
\KeywordTok{ggplot}\NormalTok{(}\DataTypeTok{data=}\NormalTok{data2)}\OperatorTok{+}\KeywordTok{geom_histogram}\NormalTok{(}\DataTypeTok{mapping =} \KeywordTok{aes}\NormalTok{(}\DataTypeTok{x =}\NormalTok{ tot_steps),}\DataTypeTok{na.rm=}\OtherTok{TRUE}\NormalTok{)}
\end{Highlighting}
\end{Shaded}

\begin{verbatim}
## `stat_bin()` using `bins = 30`. Pick better value with `binwidth`.
\end{verbatim}

\includegraphics{week-2-assignment_files/figure-latex/unnamed-chunk-2-1.pdf}

\paragraph{3)Mean and median number of steps taken each
day}\label{mean-and-median-number-of-steps-taken-each-day}

\begin{Shaded}
\begin{Highlighting}[]
\NormalTok{data3 <-}\StringTok{ }\NormalTok{data }\OperatorTok\StringTok{ }\KeywordTok{group_by}\NormalTok{(date) }\OperatorTok\StringTok{ }
\StringTok{        }\KeywordTok{summarise}\NormalTok{(}\DataTypeTok{mean_steps =} \KeywordTok{mean}\NormalTok{(steps,}\DataTypeTok{na.rm =} \OtherTok{TRUE}\NormalTok{),}
                  \DataTypeTok{med_steps =} \KeywordTok{median}\NormalTok{(steps,}\DataTypeTok{na.rm =} \OtherTok{TRUE}\NormalTok{))}
\NormalTok{data3}
\end{Highlighting}
\end{Shaded}

\begin{verbatim}
## # A tibble: 61 x 3
##    date       mean_steps med_steps
##    <chr>           <dbl>     <dbl>
##  1 2012-10-01    NaN            NA
##  2 2012-10-02      0.438         0
##  3 2012-10-03     39.4           0
##  4 2012-10-04     42.1           0
##  5 2012-10-05     46.2           0
##  6 2012-10-06     53.5           0
##  7 2012-10-07     38.2           0
##  8 2012-10-08    NaN            NA
##  9 2012-10-09     44.5           0
## 10 2012-10-10     34.4           0
## # ... with 51 more rows
\end{verbatim}

\paragraph{4)Time series plot of the average number of steps
taken}\label{time-series-plot-of-the-average-number-of-steps-taken}

\begin{Shaded}
\begin{Highlighting}[]
\NormalTok{data3}\OperatorTok{$}\NormalTok{date <-}\StringTok{ }\KeywordTok{as.Date}\NormalTok{(data3}\OperatorTok{$}\NormalTok{date)}
\KeywordTok{ggplot}\NormalTok{(}\DataTypeTok{data=}\NormalTok{data3,}\DataTypeTok{mapping =} \KeywordTok{aes}\NormalTok{(}\DataTypeTok{x=}\NormalTok{date,}\DataTypeTok{y=}\NormalTok{mean_steps))}\OperatorTok{+}\KeywordTok{geom_line}\NormalTok{(}\DataTypeTok{na.rm=}\OtherTok{TRUE}\NormalTok{)}
\end{Highlighting}
\end{Shaded}

\includegraphics{week-2-assignment_files/figure-latex/unnamed-chunk-4-1.pdf}

\paragraph{5)The 5-minute interval that, on average, contains the
maximum number of
steps}\label{the-5-minute-interval-that-on-average-contains-the-maximum-number-of-steps}

\begin{Shaded}
\begin{Highlighting}[]
\NormalTok{data4 <-}\StringTok{ }\NormalTok{data }\OperatorTok\StringTok{ }\KeywordTok{group_by}\NormalTok{(interval) }\OperatorTok\StringTok{ }
\StringTok{        }\KeywordTok{summarise}\NormalTok{(}\DataTypeTok{max_steps =} \KeywordTok{max}\NormalTok{(steps,}\DataTypeTok{na.rm=}\OtherTok{TRUE}\NormalTok{))}
\end{Highlighting}
\end{Shaded}

\paragraph{6)Code to describe and show a strategy for imputing missing
data}\label{code-to-describe-and-show-a-strategy-for-imputing-missing-data}

\begin{Shaded}
\begin{Highlighting}[]
\KeywordTok{any}\NormalTok{(}\KeywordTok{is.na}\NormalTok{(data}\OperatorTok{$}\NormalTok{steps))}
\end{Highlighting}
\end{Shaded}

\begin{verbatim}
## [1] TRUE
\end{verbatim}

\begin{Shaded}
\begin{Highlighting}[]
\NormalTok{data}\OperatorTok{$}\NormalTok{steps[}\KeywordTok{which}\NormalTok{(}\KeywordTok{is.na}\NormalTok{(data}\OperatorTok{$}\NormalTok{steps))] <-}\StringTok{ }\KeywordTok{mean}\NormalTok{(data}\OperatorTok{$}\NormalTok{steps,}\DataTypeTok{na.rm=}\OtherTok{TRUE}\NormalTok{)}
\KeywordTok{any}\NormalTok{(}\KeywordTok{is.na}\NormalTok{(data}\OperatorTok{$}\NormalTok{steps))}
\end{Highlighting}
\end{Shaded}

\begin{verbatim}
## [1] FALSE
\end{verbatim}

\paragraph{7)Histogram of the total number of steps taken each day after
missing values are
imputed}\label{histogram-of-the-total-number-of-steps-taken-each-day-after-missing-values-are-imputed}

\begin{Shaded}
\begin{Highlighting}[]
\NormalTok{data5 <-}\StringTok{ }\NormalTok{data }\OperatorTok\StringTok{ }\KeywordTok{group_by}\NormalTok{(date) }\OperatorTok\StringTok{ }\KeywordTok{summarize}\NormalTok{(}\DataTypeTok{tot_steps =} \KeywordTok{sum}\NormalTok{(steps))}
\end{Highlighting}
\end{Shaded}

\paragraph{8)Panel plot comparing the average number of steps taken per
5-minute interval across weekdays and
weekends}\label{panel-plot-comparing-the-average-number-of-steps-taken-per-5-minute-interval-across-weekdays-and-weekends}

\begin{Shaded}
\begin{Highlighting}[]
\CommentTok{#making data weekend and weekdays wise}
\NormalTok{data}\OperatorTok{$}\NormalTok{days <-}\StringTok{ }\KeywordTok{weekdays}\NormalTok{(}\KeywordTok{as.Date}\NormalTok{(data}\OperatorTok{$}\NormalTok{date))}
\NormalTok{weekend <-}\StringTok{ }\KeywordTok{grep}\NormalTok{(}\StringTok{"Saturday|Sunday"}\NormalTok{,}
\NormalTok{                data}\OperatorTok{$}\NormalTok{days,}\DataTypeTok{ignore.case =} \OtherTok{TRUE}\NormalTok{)}
\NormalTok{weekend_data <-}\StringTok{ }\NormalTok{data[weekend,]}
\NormalTok{weekend_data}\OperatorTok{$}\NormalTok{weekday <-}\StringTok{ "weekend"}
\NormalTok{weekday_data <-}\StringTok{ }\KeywordTok{subset}\NormalTok{(data,data}\OperatorTok{$}\NormalTok{days }\OperatorTok{!=}\StringTok{ }\NormalTok{weekend)}
\NormalTok{weekday_data}\OperatorTok{$}\NormalTok{weekday <-}\StringTok{ "weekday"}
\NormalTok{weekend_day <-}\StringTok{ }\KeywordTok{rbind}\NormalTok{(weekend_data,weekday_data)}
\CommentTok{# for plotting}
\NormalTok{data7 <-}\StringTok{ }\NormalTok{weekend_day }\OperatorTok\StringTok{ }\KeywordTok{group_by}\NormalTok{(interval,weekday) }\OperatorTok\StringTok{ }\KeywordTok{summarize}\NormalTok{(}\DataTypeTok{mean_steps=}\KeywordTok{mean}\NormalTok{(steps))}
\end{Highlighting}
\end{Shaded}

\begin{verbatim}
## `summarise()` has grouped output by 'interval'. You can override using the `.groups` argument.
\end{verbatim}

\begin{Shaded}
\begin{Highlighting}[]
\KeywordTok{ggplot}\NormalTok{(data7,}\DataTypeTok{mapping =} \KeywordTok{aes}\NormalTok{(}\DataTypeTok{x=}\NormalTok{interval,}\DataTypeTok{y=}\NormalTok{mean_steps))}\OperatorTok{+}\KeywordTok{geom_point}\NormalTok{()}\OperatorTok{+}\KeywordTok{facet_grid}\NormalTok{(weekday}\OperatorTok{~}\NormalTok{.)}
\end{Highlighting}
\end{Shaded}

\includegraphics{week-2-assignment_files/figure-latex/unnamed-chunk-9-1.pdf}

\end{document}
